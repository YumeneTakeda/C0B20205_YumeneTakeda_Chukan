\documentclass[uplatex]{jsarticle}
%\documentclass[a4j]{ujarticle}
\usepackage{resume}  % resume用スタイル
\usepackage{udline}  % 下線用
\usepackage{comment} % 複数行コメント

\pagestyle{plain}

\begin{document}
\twocolumn[
    \beginheader{平成31年度 コンピュータサイエンス学部 中間発表}{2019}{10}{24}{井上 研究室}
    \title{\LaTeX の復習(2)}
    \author{C0117999 井上 亮文 (Akifumi Inoue)}
    \endheader
]
\vspace{3mm}

%%ページ番号
%\setcounter{page}{11}

\section{はじめに}

井上研では,卒業論文や修士論文,それらの学内発表時に配布するレジュメ等
を\LaTeX で執筆する.このサンプルでは,\LaTeX の存在そのものを忘れた学
生のために,\LaTeX の基本を軽く復習する.また,井上研流の図や表の書き方,
井上研フォーマット独自のコマンド等についても解説する.以降の説明は,こ
のファイルのソースも参照しながら読んで欲しい.

\section{前提}

\begin{itemize}
\item 処理系 \texttt{uplatex, upbibtex}
\item コンパイル等 \texttt{latexmk}
\item 句点「.」読点「,」
\end{itemize}

\section{段落}

本文は段落に分けて執筆する.
段落とは,長い文章の中で,一つの主題をもってまとまった部分のことを指す.
\LaTeX で新しい段落を作成する場合,直前に空行を1行以上挿入すればよい.

これは1つ目の段落である.現在のテンプレートでは,段落の先頭は自動的に
字下げされる.どうしても字下げをしたくない場合,段落の先頭に
\verb|\noindent| を書く.

% ここに空行が入っている

これは2つ目の段落である.
段落の中では,空行を挿入さえしなければ,改行位置は任意であ
る.
特に説明をする内容がないので,適当な文章を書いておく.

文章中に\verb|\\|を挿入すると,\\
このように\footnote{このサンプルの \LaTeX ソースを見てみよう}
強制的に改行をすることができるが,
字下げがされない.「\verb|\\|」はあくまで改行であって,
段落ではない.
\textbf{どうしても必要な場合を除いて使わないこと.}

\section{フォント}

文字に何らかの装飾を加えたい場合は,\tabref{tab:fonts}にあるコマンドを
利用する.

\underline{下線は改行に弱い}.行をまたいで下線を入れたり,線の種類を変
更したりしたい場合は,独自コマンドの利用が必須である.

よく使うのは強調に用いる\textbf{太字}だが,使いすぎると目立たなくなって
しまう.本当に重要なところだけに利用するよう心がけよう.

等幅フォントは,プログラムやコマンドの表記に利用することが多い.
情報系の人の多くは,プロポーショナルフォントで出力されたソースコードを
見ると憤死すると言われる.\texttt{public static void main(String[] args)}と
public static void main(String[] args) のどちらが目に
やさしいかは一目瞭然だろう.

フォントサイズの変更コマンドは,表に示した以外にも多数存在する.
実際の変更は,表内部の文字が大きすぎて境界からはみ出してしまう場合に
小さくする方向で利用することがほとんどである.
本文中のフォントサイズの変更は避けること.

\begin{table}
  \centering
  % \scriptsize
  \caption{フォントの変更に用いるコマンド}
  \label{tab:fonts}
  \begin{tabular}[t]{l|l|l}
    \hline
    \hline
    コマンド & 意味 & 例\\
    \hline
    なし & 通常フォント & Dark Soul\\
    \verb|\underline{}| & 下線 & \underline{Dark Soul}\\
    \verb|\textbf{}| & 太字 & \textbf{Dark Soul}\\
    \verb|\texttt{}| & 等幅 & \texttt{Dark Soul}\\
    \verb|\textit{}| & イタリック & \textit{Dark Soul}\\
    \verb|\textsl{}| & スラント & \textsl{Dark Soul}\\
    \verb|\textsf{}| & サンセリフ & \textsf{Dark Soul}\\
    \hline
    \verb|\tiny{}| & 最小サイズ & \tiny{進捗}\\
    \verb|\scriptsize{}| & 2番めに小さい& \scriptsize{進捗}\\
    \hline
  \end{tabular}
\end{table}

\section{図}

図の挿入には \texttt{figure} 環境の中で \verb|\includegraphics| を
用いる.
\texttt{[h]} の部分には図を挿入する位置とその優先順位を指定する.
2カラムのレジュメでは \texttt{[ht]} もしくは \texttt{[tb]},
シングルカラムの卒論等では \texttt{[tb]} がよい.

\verb|\includegraphics|を\verb|\fbox{}|で囲むと,図形の周囲に枠線を書く
ことができる(\figref{fig:raster}).図どうしの境界線をはっきりさせたい場合に
用いると良いだろう.

\begin{figure}[tb]
  \centering
  \fbox{
    \includegraphics[width=3cm]{fig/leo.png}
  }
  \caption{ラスター画像の例.テキストの選択ができない.}
  \label{fig:raster}
\end{figure}

\begin{figure}[tb]
  % width や height で絶対的な大きさ指定をすることもできる
  \centering
  \includegraphics[width=0.4\linewidth]{fig/leo.pdf}
  \caption{ベクター画像の例.テキストの選択ができる.}
  \label{fig:vector}
\end{figure}

最新の \LaTeX\footnote{TeX Live 2015以上}を使っていれば,\LaTeX へ挿入
する図の形式はなんでもよい.実際,このサンプル
の\figref{fig:raster}はJPEG形式,\figref{fig:vector}はPDF形式の図であ
る.ここは大学であり,ビジネスにありがちな過去のしがらみなどないのだから,
最新で快適な環境をさっさと導入しよう.

四角形や丸などを用いた(写真以外の)図形は,
可能であればPDFやEPSなどのベクター形式で作成しよう.
ベクター形式で保存しておけば,\figref{fig:vector}のように
PDF上でテキストの選択や検索ができる他,\figref{fig:compare}の
ように拡大縮小処理にも強い.

\begin{figure}[tb]
  \centering
  \includegraphics[width=0.6\linewidth]{fig/compare.pdf}
  \caption{拡大時におけるラスター形式(左)とベクター形式(右)の比較}
  \label{fig:compare}
\end{figure}

\section{表}

表は \verb|table| 環境の中に(1)キャプション,(2)ラベル,(3)\verb|tabular|環境,
の3つを入れて作成する.

井上研では,表の書き方に細かい指定がある.
表のタイトルを指定する \verb|\caption{}| は \texttt{tabular} の上に書くこと.
表の罫線は \tabref{tab:menu} のように最上部が二重線,列の見出しと要素の間は
一重線であり,要素間および表の左右には罫線を引かない.

\begin{table}[b]
  \centering
  \footnotesize
  \caption{井上研食料庫のメニュー}
  \label{tab:menu}
  \begin{tabular}[h]{l|r|l}
    \hline
    \hline
    商品名 & 価格(円) & 備考 \\
    \hline
    水 & 30 & \\
    緑茶 & 70 & \\
    HARIBO & 30 & 2つ\\
    \hline
  \end{tabular}
\end{table}

\section{独自コマンド}

\subsection{ \texttt{udline.sty} による下線}

プリアンブルに \verb|\usepackage{udline}| と書けば,同梱され
た \texttt{udline.sty} にある様々な下線コマンドを利用できるようになる
(\tabref{tab:udline}).\uc{二重線}や\Sl{打ち消し線}も引ける.\ul{複数
  行にわたる下線も引けるようになるが,使いすぎは可読性を妨げるので,}ほ
どほどがよい.


\begin{table}[h]
  \centering
  \small
  \caption{udline.styが提供するコマンド群}
  \label{tab:udline}
  \begin{tabular}[h]{l|p{4zw}|p{4zw}|p{4zw}|p{4zw}}
    \hline
    \hline
     & 一本線・改行不可 & 飾り線・改行不可 & 一本線・改行可 & 飾り線・改行可\\
    \hline
    下線 & \verb|\unl{}| & \verb|\unc{}| & \verb|\ul{}| & \verb|\uc{}|\\
    上線 & \verb|\ovl{}| & \verb|\ovc{}| & \verb|\ol{}| & \verb|\oc{}|\\
    打ち消し線 & \verb|\sol{}| & \verb|\soc{}| & \verb|\Sl{}| & \verb|\Sc{}|\\
    \hline
  \end{tabular}
\end{table}

\subsection{図表の参照}

\LaTeX で図表を参照するときは \verb|\ref{}|を用いるが,このサンプルでは
図の参照に \verb|\figref{}|を,表の参照に \verb|\tabref|を用いている.
これら独自コマンドを用いると,図表の最初の参照部分だけが太字で表示され,
2回目以降の参照では通常フォントで表示される.

\subsection{コメント}

\LaTeX では \verb|%| 以降の文字列はコメントとして扱われる\cite{翔悟17}.
複数行をコメントアウトするには,各行の先頭に \verb|%| を打つ必要がある.

このサンプルではプリアンブルで\verb|comment|パッケージを読み込んである.
これにより, \verb|\begin{comment}|と\verb|end{comment}|で囲まれた範囲
がコメントとして扱われるようになる.

\subsection{その他}

\texttt{okumacro}を使えば丸数字 \MARU{23} や \return も出せる.
\ruby{進捗}{ゼロ}のようにルビも振ることができるが,これらが利用できるのは
\texttt{(u)platex}系のエンジンに限られる.
\texttt{xelatex}等では\texttt{bx...}パッケージを利用しよう.

% biber + biblatex は日本語の問題で諦めた.従来通り bibtex を使う.
\bibliographystyle{junsrt}    % 参考文献形式を junsrt に
\bibliography{ref}            % 参考文献DBファイルを ref.bib に

\end{document}
